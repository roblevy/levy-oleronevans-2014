\documentclass[a4paper]{article}
\usepackage{amsfonts}
\usepackage{amsmath}
\usepackage{amsthm}
\usepackage{graphicx}
\usepackage{array}
\usepackage[footnotesize,bf]{caption}
\usepackage{mathtools}
\usepackage{booktabs}
\usepackage[format=hang]{caption}
\usepackage[font=footnotesize]{subcaption}
\usepackage{varioref}
\captionsetup{justification=justified, singlelinecheck=true}
\usepackage[backend=biber,url=false,doi=false,style=authoryear]{biblatex}
\addbibresource{D:/Rob/Dropbox/PhD/Writing/bibtex/Zotero - IO Modelling.bib}
\usepackage{authblk}
\usepackage{attrib}
%\bibliographystyle{elsarticle-harv}
%\nocite{*}

\title{A global inter-country economic model based on linked input-output models}
\author[*]{Robert G. Levy}
\author[**]{Thomas P. Ol\'{e}ron Evans}
\author[*]{Alan G. Wilson}

\affil[*]{Centre for Advanced Spatial Analysis, UCL Bartlett Faculty of the Built Environment,
90 Tottenham Court Road, London W1T 4TJ, UK}
\affil[**]{Department of Mathematics, University College London, Gower Street, London WC1E 6BT, UK}


\begin{document}
\maketitle

\begin{abstract}
A global model is presented that can be used as the basis for assessing the impacts of future changes in trade, migration, security and development aid.
The model is based on input-output models for 40 countries, linked with trade data at the sector level.
This is made possible by the World Input-Output Database, a collection of input-output tables for 40 countries across 15 years, and by databases of commodities and services trade from the UN.
The model is constructed using a minimum number of assumptions, and is based as far as possible on empirical observation.
Some initial analysis of the model and its properties are also presented
\end{abstract}

\section{Introduction}
The objective of this paper is to present global economic model that can be used as the basis for assessing the impacts of future changes in trade, migration, security and development aid.
The model presented here represents a first `proof of concept' step towards this ambitious goal.
The economies of individual countries are represented as 35-sector input-output models each of which is linked through trade flows representing imports and exports.
This has recently been made feasible by the publication of the World Input-Output Database (WIOD) \parencite{Timmer2012}, a collection of input-output tables (IOTs) for 40 countries across 15 years, from 1995 to 2009.
The IOTs are linked through data from the UN covering trade in both goods\footnote{comtrade.un.org/db} and services\footnote{unstats.un.org/unsd/servicetrade/}.

The remainder of this paper is structured as follows: 
Section \ref{sec:litreview} gives an overview of existing work in this area.
Section \ref{sec:system} gives a description of the present system and outlines how data is used to calibrate the parameters of the model.
The algorithm used to calculate the output of the model is described in section \ref{sec:algorithm} and some preliminary results are given in section \ref{sec:results}.

Some concluding comments are added in section \ref{sec:conclusions}.

\section{Existing Global Economic Models} \label{sec:litreview}
In the mid-1970s, the creator of input-output economics, Wassily Leontief, had just won the Nobel prize for Economics and took the prestige that this bestowed on him as an opportunity to announce a very ambitious project to model the global economy:

\begin{quotation}
Major efforts are underway to construct a data base for a systematic input-output study not of a single national economy but of the world economy viewed as a system composed of many interrelated parts [...]
Preliminary plans provide for a description of the world economy in terms of twenty-eight groups of countries, with about forty-five productive sectors for each group. \attrib{\cite{Leontief1974}}
\end{quotation}

Some 20 years later, Faye Duchin, a former research assistant of Leontief, describes how Leontief's efforts in this area have largely been ignored by economists, describing the departures from the standard, neoclassical modelling in terms of price elasticity and elasticity of substitution as being ``too great to ignore'' \parencite{Duchin2004}. 
See section \ref{sec:iots} for more on this subject.

In the years since Leontief published his global model, Input-output analysis has been largely restricted to regional studies, of which \textcite{Akita1993}, \textcite{Khan1999} and \textcite{Luo2013a} are examples; and studies related to energy and the environment, such as  \textcite{Leontief1970}, \textcite{Joshi1999}, \textcite{Bergh2002} and \textcite{Hendrickson2006}.

However, much more recently, attention has returned to input-output modelling in a global context more generally. \textcite{Tukker2013} describe how several multi-regional input-output (MRIO) models have been developed in the very recent literature.
These are, along with the WIOD which is used in the present model, EORA \parencite{Lenzen2012}, EXIOPOL \parencite{Tukker2013a} and the slightly more mature, and proprietorial GTAP \parencite{Walmsley2012}.

\section{The System Description} \label{sec:system}
\begin{table}
\begin{center}
\begin{tabular}{cl}
$f_s^{(i)}$ & final demand on sector $s$ in country $i$\\
$n_s^{(i)}$ & investment of sector $s$ in country $i$\\
$e_s^{(i)}$ & export of sector $s$ in country $i$\\
$z_{s,t}^{\dagger(i)}$ & intermediate demand on domestic sector $s$ by sector $t$ in country $i$\\
$z_{s,t}^{*(i)}$ & intermediate demand on import sector $s$ by sector $t$ in country $i$
\end{tabular}
\end{center}
\caption{Quantities from data which define a country's economy}\label{tbl:cvars}
\end{table}

The present model has $c$ countries, the economies of which are divided into $s$ productive sectors.
We assume that each sector produces a single good, thus `sector' and `product' are used interchangeably. 
Goods produced domestically are labelled with a dagger superscript ($\dagger$) and imported goods with an asterisk superscript.
Table \ref{tbl:cvars} shows the quantities, taken from data published by the WIOD, which characterise a country's economy for a particular year.
Note that for clarity, no time subscript is added. In future time-series analyses such a subscript would have to be added.

\subsection{Input-Output Tables} \label{sec:iots}
Input-output is, at its heart, an accounting methodology. By simple summation total production of sector $s$ in country $i$ can be defined as the sum of all intermediate supply, plus supply to final demand, investment and export:
\begin{equation}\label{eqn:x}
x_s^{(i)}=\sum\limits_{t}z_{s,t}^{\dagger(i)} + f_s^{\dagger} + n_s^{\dagger} + e_s
\end{equation}
and the total import of sector $s$ in country $i$ as the sum of all intermediate supply by imported goods, plus demand for and investment of imported goods:
\begin{equation}\label{eqn:m}
m_s^{(i)}=\sum\limits_{t}z_{s,t}^{*(i)} + f_s^* + n_s^*
\end{equation}
By assembling the data values describing a country in a particular arrangement, an input-output table can be constructed, $\boldsymbol{T}^{(i)}$, as described by \textcite{Miller1985}.
Neglecting the $(i)$ superscript for clarity, the input-output table (IOT) is defined as follows:

\begin{equation}\label{eqn:T}
\begin{array}{rcc}
\boldsymbol{T} & = & \begin{array}{rc}
\begin{array}{cc} \hspace*{17mm} & \mbox{Sector} \end{array} & \begin{array}{ccccccc} 1 & \mathellipsis & s & \mbox{F.D.} & \mbox{Inv} & \mbox{Exp} & \mbox{Tot} \end{array} \vspace*{2mm} \\
\begin{array}{r}
\begin{array}{rc}
\mbox{Domestic} & \left\{ \begin{array}{c}
1 \\
\vdots \\
s
\end{array} \right. \\
\mbox{Imports} & \left\{ \begin{array}{c}
1 \\
\vdots \\
s \\
\end{array} \right.
\end{array} \vspace*{2mm} \\
%\begin{array}{cc} \hspace*{17mm} & \mbox{Value Added} \end{array}
\end{array} &
\left( \begin{array}{ccccccc}
z^{\dag}_{1,1} & \mathellipsis & z^{\dag}_{1,s} & f^\dag_{1} & n^\dag_{1} & e^\dag_{1} & x_1 \\
\vdots & \ddots & \vdots & \vdots & \vdots & \vdots & \vdots \\
z^{\dag}_{s,1} & \mathellipsis & z^{\dag}_{s,s} & f^\dag_{s} & n^\dag_{s} & e^\dag_{s} & x_{s} \\
z^*_{1,1} & \mathellipsis & z^*_{1,s} & f^*_{1} & n^*_{1} & 0 & i_1 \\
\vdots & \ddots & \vdots & \vdots & \vdots & \vdots & \vdots \\
z^*_{s,1} & \mathellipsis & z^*_{s,s} & f^*_{s} & n^*_{s} & 0 & i_{s} \vspace*{2mm} \\
%v_{1} & \mathellipsis & v_{s} & 0 & 0 & 0 & G
\end{array} \right)
\end{array}
\end{array}
\end{equation}

\noindent It will often be convenient to gather the quantities in table \ref{tbl:cvars} with a single subscript into vectors, and those with two subscripts into matrices. 
We can then characterise a country's economy through the $s$-vectors $\boldsymbol{f}^{(i)}$, $\boldsymbol{n}^{(i)}$ and $\boldsymbol{e}^{(i)}$, and by the $s\times s$ matrices $\boldsymbol{Z}^{\dagger(i)}$ and $\boldsymbol{Z}^{*(i)}$.
In matrix form, $\boldsymbol{T}$ may be written:

\begin{equation}\label{eqn:Tvectorised}
\begin{array}{rcc}
\boldsymbol{T} & = & 
\left(
	\begin{array}{ccccccc}
 & \boldsymbol{Z}^{\dag} & & \boldsymbol{f}^\dag & \boldsymbol{n}^\dag & \boldsymbol{e} & \boldsymbol{x} \\
 & \boldsymbol{Z}^* & & \boldsymbol{f}^* & \boldsymbol{n}^* & \boldsymbol{0} & \boldsymbol{i} \\
	\end{array} 
\right)
\end{array}
\end{equation}

\subsection{A Country Model}\label{sec:countries}
In the standard input-output model, each country is described by the input-output table described in section \ref{sec:iots} above.
From the elements of $\boldsymbol{Z}^\dagger$ and $\boldsymbol{Z}^*$, each sector has a complete `recipe' for making its output, in terms of the quantities of each good used as input, both domestic and imported.
\subsubsection*{Technical Coefficients}\label{sec:techcoeffs}
By dividing by total output, we can arrive at a set of \textit{technical coefficients} which define the input required per unit output.
The amount of good $r$ required by sector $s$ to produce a single unit of output is thus:
\begin{equation}\label{eq:adagger}
a_{r,s}^\dagger = \frac{x_s}{z^\dagger_{r,s}}
\end{equation}
for domestically produced $r$, and
\begin{equation}\label{eq:astar}
a_{r,s}^* = \frac{x_s}{z^*_{r,s}}
\end{equation}
for imported $r$. There are therefore $2s \times s$ technical coefficients for each country\footnote{Recall that we are assuming a single year throughout this treatment.
In time series analyses, data for which are indeed available from the WIOD, there will be $2s \times s$ technical coefficients for every country in every year.}.
These technical coefficients then allow the input requirements (both domestic and imported) to be calculated for any exogenously given vector of final, investment and export demands. For sector $s$, allowing $f$ to include all three types of demand for notational simplicity:
$$
x_s = f^\dagger_s + \sum_r{a^\dagger_{s,r}x_r}
$$
or, in matrix representation:
\begin{align}
\boldsymbol{x}& = \boldsymbol{f^\dagger} 
+ \boldsymbol{A^\dagger}\boldsymbol{x} \nonumber\\
\boldsymbol{x}& = (\boldsymbol{I} - \boldsymbol{A^\dagger})^{-1} 
\boldsymbol{f^\dagger}\label{eqn:xIRIO}
\end{align}
and then
\begin{equation}\label{eqn:mIRIO}
\boldsymbol{m} = \boldsymbol{f^*} 
+ \boldsymbol{A^*}\boldsymbol{x}
\end{equation}
thus the domestic total production and the imports are completely determined from demand and the technical coefficients.

\subsubsection*{Import Ratios}\label{sec:importratios}
Since the goal of this model is to represent all the countries in the world, many of the country IOTs will have to be estimated from available data (in fact, all except the 40 of the WIOD will).
It is therefore crucial that the model be as parsimonious as possible in terms of parameters.
To this end, the model features two simplifications inspired by the description of Leontief's global model given by \textcite{Duchin2004}.
Leontief assumed that engineers in an importing country do not care where a product originated from; they will simply know that domestic production does not meet their demand, and instead demand a perfectly-substitutable imported good.

In a similar spirit, when a product in the present model arrives at the shores of an importing country, it enters a pool with domestically produced goods, at which point the two become indistinguishable\footnote{Note that this concept is referred to in \textcite{Miller1985} as \textit{import similarity}}.
The only thing which is specified by the model is the ratio of imported to domestic goods in this pool, which remains fixed. This is called the \textit{import ratio}, and is calculated as:
\begin{equation}\label{eqn:importratio}
d_s^{(i)} = \frac{m_s^{(i)}}{x_s^{(i)} + m_s^{(i)}}
\end{equation}
where $x_s^{(i)}$ and $m_s^{(i)}$ are the total production and import of sector $s$, calculated via equations \eqref{eqn:x} and \eqref{eqn:m} respectively.

This simplification makes easier the job of estimating countries for which the WIOD has no data, since it reduces the number of technical coefficients to $s \times s$.
This reduced set of technical coefficients means that only total inter-sector flows, $\boldsymbol{Z}$, equivalent to $\boldsymbol{Z}^{\dagger} + \boldsymbol{Z}^{*}$ in equation \eqref{eqn:Tvectorised}, need to be estimated, along with an $s$-vector, $\boldsymbol{d}$, of import ratios. 
In a similar spirit to equation \eqref{eq:adagger}, the technical coefficients can then be calculated as
\begin{equation}
a_{r,s} = \frac{x_s}{z_{r,s}}
\end{equation}
allowing the calculation of $\boldsymbol{x}$ as per equation \eqref{eqn:xIRIO}:
\begin{equation}
\boldsymbol{x} = 
(\boldsymbol{I} - 
(\boldsymbol{I} - \boldsymbol{\hat{d}})
\boldsymbol{A})^{-1} 
(\boldsymbol{I} - \boldsymbol{\hat{d}})\boldsymbol{f}\label{eqn:xmodel}
\end{equation}
and hence of $\boldsymbol{m}$ from equation \eqref{eqn:importratio}:
\begin{equation}
\boldsymbol{m} = 
(\boldsymbol{I} - 
\boldsymbol{\hat{d}})^{-1} 
\boldsymbol{\hat{d}}\boldsymbol{x}\label{eqn:mmodel}
\end{equation}
where $\boldsymbol{\hat{d}}$ is an $s \times s$ matrix whose diagonal elements are the import ratios, $d_s$, and $\boldsymbol{I}$ is the identity matrix.

Also due to this assumption, only $s$ elements of final demand and investment need to be estimated, rather than the $2s$ elements shown in equation \eqref{eqn:T}.
In a 35 sector model, such as that used by the WIOD, this reduces the number of flows to be estimated from $70 \times 37 = 2590$ (37 being the 35 sector columns plus $f$ and $n$) to $35 \times 38 = 1330$ (the columns plus $f$, $n$ and the import ratios).

\subsection{An International Trade Model}\label{sec:trade}
In standard inter-regional input-output modelling (IRIO), each sector in each country is explicit about which countries it gets its imports from. 
This requires each sector to have $s \times c$ technical coefficients.
This is represents a huge data requirement, but one which the WIOD have fulfilled in the world input-output tables (WIOTs) which they publish.

In order to address this onerous data requirement and make more plausible the inclusion of countries for which the WIOD publishes no data, a second assumption is made related to what Leontief via \textcite{Duchin2004} refers to as Export Shares.

\subsubsection*{Import Propensities}
The assumption is that each country gets each product from each other country in the world in fixed proportions.
We refer to these fixed proportions as \textit{import propensities} as they describe a country's propensity to import from each other country. The propensity for country $j$ to import good $s$ from country $i$ is given by
\begin{equation}
p^{(i,j)}_s = \frac{y^{(i,j)}_s}{\sum_k{y^{(k,j)}_s}}
\end{equation}
where $y^{(i,j)}_s$ is the trade flow of good $s$ from country $i$ to country $j$. The $y_s$ are taken from the UN COMTRADE database, with a mapping from 6-figure product code to sector kindly provided by the WIOD team.

Given the import requirements of each country from by equation \eqref{eqn:mmodel}, the export demand on sector $s$ in country $i$ due to demand from country $j$ is given by:
\begin{equation*}
e_s^{(i,j)} = p_s^{(i,j)}m_s^{(j)}
\end{equation*}
and total export demand in country $i$ is therefore:
\begin{equation}\label{eqn:emodel}
e_s^{(i)} = \sum_k{p_s^{(i,k)}m_s^{(k)}}
\end{equation}
\noindent Equations \eqref{eqn:importratio}--\eqref{eqn:emodel} thus describe a system which defines the total productions, $x_s^{(i)}$ of all sectors in all countries, and every trade flow $y_s^{(i,j)}$ given a set of technical coefficients, $a_{r,s}^{(i)}$, import ratios, $d_s^{(i)}$, trade propensities, $d_s^{(i,j)}$, and final demands\footnote{In the first iteration of the model investments, $n_s^{(i)}$ are ignored. Data is provided by the WIOD for this but is not used in this iteration of the model.}, $f_s^{(i)}$.


\section{The Model Algorithm}\label{sec:algorithm}

\section{Results}\label{sec:results}

\section{Conclusions}\label{sec:conclusions}

\printbibliography

\end{document} 
