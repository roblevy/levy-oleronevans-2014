\documentclass[a4paper]{article}
\usepackage{amsfonts}
\usepackage{amsmath}
\usepackage{amsthm}
\usepackage{graphicx}
\usepackage{array}
\usepackage[footnotesize,bf]{caption}
\usepackage{mathtools}
\usepackage{booktabs}
\usepackage[format=hang]{caption}
\usepackage[font=footnotesize]{subcaption}
\usepackage{varioref}
\captionsetup{justification=justified, singlelinecheck=true}
\usepackage[backend=biber,url=false,style=authoryear]{biblatex}
\addbibresource{D:/Rob/Dropbox/PhD/Writing/bibtex/library.bib}
\usepackage{authblk}
%\bibliographystyle{elsarticle-harv}
%\nocite{*}

\title{A global inter-country economic model based on linked input-output models}
\author[*]{Robert G. Levy}
\author[**]{Thomas P. Ol\'{e}ron Evans}
\author[*]{Alan G. Wilson}

\affil[*]{Centre for Advanced Spatial Analysis, UCL Bartlett Faculty of the Built Environment,
90 Tottenham Court Road, London W1T 4TJ, UK}
\affil[**]{Department of Mathematics, University College London, Gower Street, London WC1E 6BT, UK}


\begin{document}
\maketitle

\begin{abstract}
A global model is presented that can be used as the basis for assessing the impacts of future changes in trade, migration, security and development aid.
The model is based on input-output models for 40 countries, linked with trade data at the sector level.
This is made possible by the World Input-Output Database, a collection of input-output tables for 40 countries across 15 years, and by databases of commodities and services trade from the UN.
The model is constructed using a minimum number of assumptions, and is based as far as possible on empirical observation.
Some initial analysis of the model and its properties are also presented
\end{abstract}

\section{Introduction}
The objective of this paper is to present global economic model that can be used as the basis for assessing the impacts of future changes in trade, migration, security and development aid.
The model presented here represents a first `proof of concept' step towards this ambitious goal.
The economies of individual countries are represented as 35-sector input-output models each of which is linked through trade flows representing imports and exports.
This has recently been made feasible by the publication of the World Input-Output Database (WIOD) \parencite{Timmer2012}, a collection of input-output tables (IOTs) for 40 countries across 15 years, from 1995 to 2009.
The IOTs are linked through data from the UN covering trade in both goods\footnote{comtrade.un.org/db} and services\footnote{unstats.un.org/unsd/servicetrade/}.

The remainder of this paper is structured as follows: 
Section \ref{sec:litreview} gives an overview of existing work in this area.
Section \ref{sec:system} gives a description of the present system.
Section \ref{sec:iots} input-output modelling, and the IOTs made available by the WIOD.
Sections \ref{sec:countries} and \ref{sec:trade} discuss how data are used as the basis of the country model and the trade model respectively.
The algorithm used to calculate the output of the model is described in section \ref{sec:algorithm} and some preliminary results are given in \ref{sec:results}.

Some concluding comments are added in section \ref{sec:conclusions}.

\section{Existing Global Economic Models} \label{sec:litreview}

\section{The System Description} \label{sec:system}

\section{The Input-Output Tables} \label{sec:iots}

\section{A Country Model}\label{sec:countries}

\section{An International Trade Model}\label{sec:trade}

\section{The Model Algorithm}\label{sec:algorithm}

\section{Results}\label{sec:results}

\section{Conclusions}\label{sec:conclusions}

\printbibliography

\end{document} 
