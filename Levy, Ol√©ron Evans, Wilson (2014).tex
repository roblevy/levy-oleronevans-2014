\documentclass[preprint,authoryear,3p]{elsarticle}
\usepackage{amsfonts}
\usepackage{amsmath}
\usepackage{amsthm}
\usepackage{graphicx}
\usepackage{array}
\usepackage[footnotesize,bf]{caption}
\usepackage{mathtools}
\usepackage{booktabs}
\usepackage[format=hang]{caption}
\usepackage[font=footnotesize]{subcaption}
\usepackage{varioref}

\captionsetup{justification=justified, singlelinecheck=true}

%\bibliographystyle{elsarticle-harv}
%\nocite{*}

\journal{A Journal}
\begin{document}
\begin{frontmatter}
\title{A Global Trade and Input-Output Demonstration Model (v11)}
\author[casa]{Robert G. Levy \corref{cor1}}
\ead{rob.levy.11@ucl.ac.uk}
\author[casa,uclmat]{Thomas P. Ol\'{e}ron Evans}
\ead{thomas.evans.11@ucl.ac.uk}
\author[casa]{Alan G. Wilson}
\ead{a.g.wilson@ucl.ac.uk}
\address[casa]{Centre for Advanced Spatial Analysis, UCL Bartlett Faculty of the Built Environment,
90 Tottenham Court Road, London W1T 4TJ, UK}
\address[uclmat]{Department of Mathematics, University College London, Gower Street, London WC1E 6BT, UK}
\cortext[cor1]{Corresponding author. Tel.: +44 (0)20 3108 3877; Fax: +44 (0)20 3108 3258}

\begin{abstract}
This paper introduces a new framework for the study of dynamical human and economic systems at the global level. The framework is based around two large data sets, the first a description of the internal structure of countries' economies as viewed through the lens of goods and services flows, and the second a product-level trade database recording flows of goods and services between countries. The framework is used to build a demonstration model, using a minimum number of assumptions and based as far as possible on empirical observation, which provides a first step towards a truly global model for the study of large-scale economic phenomena, including trade, migration, international security and development aid.
\end{abstract}

\begin{keyword}

\end{keyword}

\end{frontmatter}

\section{Introduction}
The analysis of networks has become a focal point for many branches of social science research. Under the umbrella term `\textit{complexity science}' new data- and computing-intensive modelling techniques are making their way from the physical sciences into sociology, for example in the study of the spreading of ideas through social networks; into economics, the study of behaviour in online auctions being a recent example; and into political science, where networks are used to study the influence of media on ethnic conflict.

But the largest-scale network of all, the global economy, remains a poorly understood system. Traditional economics approaches have relied on highly stylised models when analysing international trade, and the economies themselves have often been assumed to consist of a manufacturing sector producing a single, undifferentiated good. A majority of the literature on migration focuses on explaining and predicting the size of migration flows, and those studies which attempt to say something about the effects on the economies of the sender or recipient countries have tended to be small-scale case studies rather than treating migration as part of a single global system. A similar thing can be said of the development literature. Studies which attempt to help policy-makers target development aid have focused on the assessment of individual interventions in particular development contexts rather than trying to say something about the distribution of development aid globally.

The recent publication of country accounts describing the flow of goods and services through the production process at both national and international levels has made it possible to follow economic activity both around a single economy and around the world. In measuring the importance of certain goods and services in the production of others, an approach known as input-output modelling, we can build a picture of a global `value chain' along  which goods and services pass, gaining value as they move from stage to stage. The production of a car is a simple example of this. Iron ore is mined in economic sector $A$, and sells on the market at some price per tonne, presumed relatively small. Sector $B$ then refines the ore into iron sheeting, selling the result on the market at a markup. Sector $B$ has thus added value to the original product. The iron sheeting is used by sector $C$ to produce a finished car which sells for many times the price of the iron and other components which go into its production, representing more value added again. Input-output modelling allows us to reconstruct value chains such as these and, by combining such analyses with international trade flow data, follow goods and services not just at the sector level within a country but at the country-sector level internationally. In this way, we arrive at a picture of the flow of economic activity around the globe, and the associated value added by each step of the production process.

The picture of global value chains we have thus arrived at is based almost entirely on data. The goods and services are actually observed, by customs and revenue officials or by means of company survey, flowing around the chain and gaining value as they do so. We can use this data-driven value-chain description of the global economy as a starting point for a global model which will allow us to use the tools of network and complexity science to study questions of global dynamics such as those outlined above.

\section{Input-Output Analysis}



\subsection{Sector Aggregation}





\pagebreak





\section{Estimating IO tables for non-WIOD countries}

\subsection{Motivation and concept}

MAP SHOULD INCLUDE INDIA

A significant barrier to the creation of a global demonstration model using the WIOD tables is the fact that only 40 countries are included in the database, of which a majority (27) are EU member states. A major concern is the lack of input-output tables for African countries, whose economies might be expected to differ considerably from those of countries in other regions.

To address this issue, it is necessary to create estimated input-output tables for those countries for which the WIOD has no data. Attempting to create tables which closely match the true economies in question is not feasible, since to do so would require investigations as detailed as those conducted to construct the WIOD. Instead, we aim to produce tables which match certain key economic features of the countries in question (e.g. GDP, total value of imports, total value of exports) which can be readily obtained from existing data sources.

To do this we employ kernel principal component analysis (kPCA), a widely used technique for identifying patterns in high-dimensional data (REFERENCE). For a given year, the 40 aggregated input-output tables are treated as 160-dimensional data points and their distribution is analysed to provide information about the region in which other unknown points might lie. This method presupposes that, while each aggregated input-output table provides a 160-dimensional description of a national economy, allowing for a certain amount of noise, the space of possible or plausible national economies is actually a lower dimensional manifold embedded in this space.

To estimate the input-output table for a country which is not in the WIOD database, we choose a point that lies in the intersection of the manifold of plausible economies identified by kPCA and the region formed by constraints imposed by data which is known about the country. If the dimension of the manifold is appropriately chosen with reference to the number of constraints, the intersection will be a single point or finite set of points, provided that the relevant surfaces do not intersect tangentially and that the regions and manifold are sufficiently well-behaved (CLEARLY THIS NEEDS BETTER EXPLANATION).




\subsection{Kernel methods}




\subsection{Linear principal component analysis}



\paragraph*{Introduction}

Given points $\textbf{x}_i \in \mathbb{R}^n$, $i \in \{ 1, \mathellipsis, N \}$, principal component analysis (PCA) offers a means of understanding how they are distributed across $\mathbb{R}^n$ and is a commonly used tool in the analysis of high-dimensional data.

Suppose that, allowing for a certain amount of stochastic noise, the $\textbf{x}_i$ are not distributed throughout the whole of $\mathbb{R}^n$ but over some lower dimensional manifold embedded within it. If such a manifold can be identified, the data points can be projected on to it, thus removing noise and reducing the dimensionality of the data, potentially allowing it to be more easily analysed and understood.

In linear PCA, the simplest case, it is assumed that the data points approximately lie on an unknown hyperplane $H$ passing through the mean point $\overline{\textbf{x}}$ and that the magnitude of any noise perpendicular to $H$ is small compared to the true spread of the data. Based on these assumptions, $H$ is assumed to lie parallel to the subspace spanned by the vectors representing those directions in which the variance of the data is largest. [REFERENCE]



\paragraph*{Formal description of linear PCA}

Consider $N$ data points:
$$
\textbf{x}_i \in \mathbb{R}^n \; , \; \; i \in \{1, \mathellipsis , N \}
$$
To perform linear PCA, the data must be centred by subtracting the mean:
\begin{align*}
\textbf{y}_{i} &= \textbf{x}_{i} - \overline{\textbf{x}} \\
\overline{\textbf{x}} &= \frac{1}{N} \displaystyle \sum_{i =1}^N \textbf{x}_{i}
\end{align*}
The centred vectors are then collected into the data matrix $T$:
\begin{equation*}
T = \left(
\begin{tabular}{cccc}
$\uparrow$ & $\uparrow$ & $\mathellipsis$ & $\uparrow$\\
$\textbf{y}_{1}$ & $\textbf{y}_{2}$ & $\mathellipsis$ & $\textbf{y}_{N}$ \\
$\downarrow$ & $\downarrow$ & $\mathellipsis$ & $\downarrow$
\end{tabular} \right) \\
\end{equation*}
and the data covariance matrix $S = N^{-1} T {T}^\top$ is calculated. Since covariance matrices are positive semi-definite, the eigenvalues of $S$ are all real and non-negative and can be labelled in descending order of size:
$$\lambda_1 \geq \lambda_2 \geq \mathellipsis \geq \lambda_n \geq 0$$

The corresponding eigenvectors $\textbf{v}_1 \; , \textbf{v}_2 \; , \; \mathellipsis \; , \textbf{v}_n$ are mutually orthogonal, since $S$ is symmetric.

It transpires that the direction of maximum variance of the data in $\mathbb{R}^n$ is the eigenvector $\textbf{v}_1$ of $S$ that corresponds to the largest eigenvalue $\lambda_1$ [REFERENCE]. Similarly, the direction of maximum variance of the data in the orthogonal complement of $\text{span}(\{\textbf{v}_1\})$ is $\textbf{v}_2$, the direction of maximum variance of the data in the orthogonal complement of $\text{span}(\{\textbf{v}_1,\textbf{v}_2\})$ is $\textbf{v}_3$ and so on. Moreover each eigenvalue is equal to the variance of the data projected in the direction of the corresponding eigenvector.

The eigenvalues are normalised to provide a measure of `proportional variance' in the direction of each eigenvector:
\begin{align*}
\lambda_i^{*} = \lambda_i \left[ \displaystyle \sum_{j=1}^{n} \lambda_j \right]^{-1} \; , \; \; i \in \{1, \mathellipsis , n \}
\end{align*}

Given a particular threshold value $\epsilon \in (0,1)$, let $k$ be the smallest positive integer such that:
$$ \displaystyle \sum_{j = 1}^{k} \lambda_{j}^{*} \geq 1 - \epsilon $$

The corresponding eigenvectors $\textbf{v}_{1} , \mathellipsis , \textbf{v}_{k}$ are described as the $k$ (linear) principal components of the data. For small $\epsilon$, it can be seen that the principal components account for the majority of the variance in the data, since the total of the variances in the directions of the remaining eigenvectors as a proportion of the total of the variances in the directions of all the eigenvectors does not exceed $\epsilon$.

The hyperplane supposed to represent the true domain of the data is therefore:
$$ H = \left\{ \overline{\textbf{x}} + \displaystyle \sum_{j = 1}^{k} \alpha_j \textbf{v}_{j} : \alpha_i \in \mathbb{R} \; , \; \; i \in \{ 1, \mathellipsis, k \} \right\} $$
Letting $P$ be a matrix of the principal components:
\begin{equation*}
P = \left(
\begin{tabular}{cccc}
$\uparrow$ & $\uparrow$ & $\mathellipsis$ & $\uparrow$\\
$\textbf{v}_{1}$ & $\textbf{v}_{2}$ & $\mathellipsis$ & $\textbf{v}_{k}$ \\
$\downarrow$ & $\downarrow$ & $\mathellipsis$ & $\downarrow$
\end{tabular} \right)
\end{equation*}
$H$ can be written:
$$
H = \left\{ \overline{\textbf{x}} + P \mbox{\boldmath$\alpha$} : \mbox{\boldmath$\alpha$} \in \mathbb{R}^{k} \right\}
$$
De-noised data $\textbf{z}_i$ can be obtained by projecting any point $\textbf{x}_i$ on to $H$:
$$ \textbf{z}_i = \overline{\textbf{x}} + \displaystyle \sum_{j = 1}^{k} \left\langle \textbf{y}_i , \textbf{v}_{j} \right\rangle \textbf{v}_{j} \; , \; \; i \in \{1, \mathellipsis , N \} $$
where $\langle \bullet , \bullet \rangle$ is the standard inner product on $\mathbb{R}^n$. In matrix form, this equation becomes:
\begin{equation*}
Z = \overline{X} + PP^\top T
\end{equation*}
where $Z$ is the matrix of denoised data and $\overline{X}$ is an $n \times N$ matrix of copies of the mean data vector:
\begin{align*}
Z =& \left(
\resizebox{3.3cm}{0.9cm}{\begin{tabular}{cccc}
$\uparrow$ & $\uparrow$ & $\mathellipsis$ & $\uparrow$\\
$\textbf{z}_{1}$ & $\textbf{z}_{2}$ & $\mathellipsis$ & $\textbf{z}_{k}$ \\
$\downarrow$ & $\downarrow$ & $\mathellipsis$ & $\downarrow$
\end{tabular}} \right) \\
\overline{X} =& \left(
\resizebox{3.3cm}{0.9cm}{\begin{tabular}{cccc}
$\uparrow$ & $\uparrow$ & $\mathellipsis$ & $\uparrow$\\
$\overline{\textbf{x}}$ & $\overline{\textbf{x}}$ & $\mathellipsis$ & $\overline{\textbf{x}}$ \\
$\downarrow$ & $\downarrow$ & $\mathellipsis$ & $\downarrow$
\end{tabular}} \right) \\
\end{align*}






\subsection{Kernel principal component analysis}









\subsection{Methodology} \label{methods}

Labelling the WIOD countries $1, \mathellipsis, 40$, the IO tables may be represented as 160-dimensional vector data points:
$$
\textbf{x}_{c,t} = (x_{c,t,1}, \mathellipsis, x_{c,t,160})^\top
$$
for $c \in C = \{ 1, \mathellipsis, 40 \}$ and $t \in T = \{ 1995, \mathellipsis 2009 \}$.

To perform linear PCA, the data for each year must first be centred:
\begin{align*}
\textbf{y}_{c,t} &= \textbf{x}_{c,t} - \overline{\textbf{x}}_{t} \\
\overline{\textbf{x}}_{t} &= \frac{1}{40} \displaystyle \sum_{c \in C} \textbf{x}_{c,t}
\end{align*}

The centred vectors are then collected into data matrices for each year:
\begin{equation*}
Y_t = \left(
\begin{tabular}{cccc}
$\uparrow$ & $\uparrow$ & $\mathellipsis$ & $\uparrow$\\
$\textbf{y}_{1,t}$ & $\textbf{y}_{2,t}$ & $\mathellipsis$ & $\textbf{y}_{40,t}$ \\
$\downarrow$ & $\downarrow$ & $\mathellipsis$ & $\downarrow$
\end{tabular} \right) \\
\end{equation*}

The linear principal components of the data for each year are the eigenvectors $\{ \textbf{v}_{t}^{(1)} , \mathellipsis , \textbf{v}_{t}^{(160)} \}$ of the data covariance matrices $S_t$, defined by:
$$ S_t = \frac{1}{40} Y_t {Y_t}^\top $$
while the corresponding eigenvalues $\{ \lambda_t^{(1)} , \mathellipsis , \lambda_t^{(160)} \}$ are the variances of the data projected in the directions of the corresponding eigenvectors.

The eigenvalues are then normalised to provide the proportional variances in each component and relabelled (along with the corresponding eigenvectors) in descending order of size:
\begin{align*}
\lambda_t^{*(1)} , \mathellipsis , \lambda_t^{*(160)} \\
\textbf{v}_{t}^{*(1)} , \mathellipsis , \textbf{v}_{t}^{*(160)}
\end{align*}

Given a particular threshold value of remaining proportional variance $\epsilon \in (0,1)$, consider the minimal subset $\{ \textbf{v}_{t}^{*(1)} , \mathellipsis , \textbf{v}_{t}^{*(k)} \}$ of these vectors such that:
$$ \displaystyle \sum_{i = 1}^{k} \lambda_t^{*(i)} \geq 1 - \epsilon $$

We hypothesise that all data points representing the true national IO tables for non-WIOD countries in year $t$ will lie close to the hyperplane $H_{t}$ defined by:
$$
H_{t} = \left\{ \overline{\textbf{x}}_{t} + \displaystyle \sum_{i = 1}^{k} \alpha_i \textbf{v}_{t}^{*(i)} : \alpha_i \in \mathbb{R} \; , \; \forall i \in \{ 1 , \mathellipsis , k \} \right\}
$$
or in matrix form:
$$
H_t = \left\{ \overline{\textbf{x}}_t + P \mbox{\boldmath$\alpha$} : \mbox{\boldmath$\alpha$} \in \mathbb{R}^{k} \right\}
$$
where:
\begin{equation*}
P = \left(
\begin{tabular}{cccc}
$\uparrow$ & $\uparrow$ & $\mathellipsis$ & $\uparrow$\\
$\textbf{v}_{t}^{*(1)}$ & $\textbf{v}_{t}^{*(2)}$ & $\mathellipsis$ & $\textbf{v}_{t}^{*(k)}$ \\
$\downarrow$ & $\downarrow$ & $\mathellipsis$ & $\downarrow$
\end{tabular} \right)
\end{equation*}








\subsection{Using PCA to estimate unknown data points} \label{PCAestimation}

Consider a country $K$ that is not part of the WIOD data. Suppose we have certain information about that country for a given year $t$, which can be expressed as a set of $n$ linear constraints on the entries of that country's unknown IO table. For example, we may have data on the total production of each of the aggregated sectors contained in the table.

Let $\textbf{x}_t = (x_{t,1}, \mathellipsis, x_{t,160})^\top$ be the vector representing the IO table of country $K$ (see Section \ref{methods}) in year $t$ and let $\overline{\textbf{x}}_t$ be the mean vector of the IO tables of the WIOD countries in that year. The constraints imposed by the information that we know are:
\begin{align*}
\begin{array}{ccccl}
a_{1,1}x_{t,1} & + \mathellipsis + & a_{1,160}x_{t,160} & = & \beta_1 \\
\vdots & \vdots & \vdots & \vdots & \vdots \\
a_{n,1}x_{t,1} & + \mathellipsis + & a_{n,160}x_{t,160} & = & \beta_n
\end{array}
\end{align*} 
In matrix form, this can be written:
\begin{align} \label{constraints}
A \textbf{x}_t = \mbox{\boldmath$\beta$}
\end{align} 
where $A \in \mathbb{R}^{n \times 160}$, $\textbf{x}_t \in \mathbb{R}^{160}$ and $\mbox{\boldmath$\beta$} \in \mathbb{R}^n$.

Equation (\ref{constraints}) will not generally have a unique solution, since we expect that the amount of information available to us about country $K$ in year $t$ will not be sufficient to reconstruct a complete aggregated IO table. Certainly, this will be the case if $n < 160$, so we assume this to be the case.

Consider the principal component matrix $V_t \in \mathbb{R}^{160 \times n}$ formed from the first $n$ linear principal components of the data for year $t$:
\begin{equation*}
V_t = \left(
\begin{tabular}{cccc}
$\uparrow$ & $\uparrow$ & $\mathellipsis$ & $\uparrow$\\
$\textbf{v}_{t}^{(1)}$ & $\textbf{v}_{t}^{(2)}$ & $\mathellipsis$ & $\textbf{v}_{t}^{(n)}$ \\
$\downarrow$ & $\downarrow$ & $\mathellipsis$ & $\downarrow$
\end{tabular} \right) \\
\end{equation*}

The surface $H_t$ on which the data are supposed to lie is therefore given by:
$$
H_{t} = \left\{ \overline{\textbf{x}}_t + V_t \mbox{\boldmath$\alpha$} : \mbox{\boldmath$\alpha$} \in \mathbb{R}^n \right\}
$$

Since the columns of $V_t$ are linearly independent, $H_t$ is an $n$-dimensional hyperplane in $\mathbb{R}^{160}$. Similarly, assuming that $A$ is non-singular (i.e. assuming that each piece of information that we have about country $K$ in year $t$ is independent and that the information is not contradictory), the solution space of (\ref{constraints}) is a $(160-n)$-dimensional hyperplane in $\mathbb{R}^{160}$.

These hyperplanes will intersect in a single point, provided that the column space of $V_t$ lies in the orthogonal complement of the kernel of $A$. In this case, the point of intersection $\textbf{x}^*_t$ is taken as the estimated value of $\textbf{x}_t$.

Since $\textbf{x}^*_t \in H_{t}$, there exists some $\mbox{\boldmath$\alpha$}_0 \in \mathbb{R}^n$ such that:
\begin{align} \label{wy_estimate}
\textbf{x}^*_t = \overline{\textbf{x}}_t + V_t \mbox{\boldmath$\alpha$}_0
\end{align}
and, since $\textbf{x}^*_t$ also satisfies (\ref{constraints}), we have:
$$
A \left( \overline{\textbf{x}}_t + V_t \mbox{\boldmath$\alpha$}_0 \right) = \mbox{\boldmath$\beta$}
$$
Rearranging gives:
$$
A V_t \mbox{\boldmath$\alpha$}_0 = \mbox{\boldmath$\beta$} - A \overline{\textbf{x}}_t
$$
Having supposed that the intersection point $\textbf{x}^*_t$ is unique, there is also a unique solution for $\mbox{\boldmath$\alpha$}_0$, which can be substituted back into (\ref{wy_estimate}) to find $\textbf{x}^*_t$.





\subsection{Comments on the methodology}

The method described above is an ideal implementation of the theory, in which PCA is performed separately for each year in the WIOD database, so all data points in each calculation relate to different countries. The method therefore allows for the theoretical distribution of IO tables $\textbf{x}_{c,t}$ across $\mathbb{R}^{160}$ to be time dependent.

Unfortunately, owing to the relatively small amount of data available, overfitting of the PCA hyperplanes is a considerable danger. Indeed, the fact that the number of data points for each year (40) is smaller than the dimension of the space from which they are drawn (160) clearly poses problems of this kind, since given any $N$ points in $\mathbb{R}^n$, at least one $(N-1)$-dimensional hyperplane may be found which passes through them all.

For this reason, it was decided that the PCA calculation should be performed using all the available data points, representing the IO tables of the 40 WIOD countries across the 15 years for which data was available. This reduces the problem of overfitting, since the number of points (600) is now comfortably larger than the dimension of the space they inhabit (160). However, it raises the additional issue that the level of independence of the data is significantly reduced, since there will be strong correlations between the IO tables of a particular country from one year to the next. This means that the amount of original information provided by each data point is reduced. However, given the nature of the problem to be addressed - the estimation of unknown data points using a limited amount of available data - it is not clear that a simultaneous solution to the twin issues of overfitting and independence is possible. The method adopted is considered to represent `the lesser of two evils'.

A further issue with the method arises from the fact that it extrapolates the shape of unknown economies using the shapes of known economies. As previously noted, the WIOD countries do not constitute a representative sample of the world, since they include a preponderance of EU member states with no African countries at all. If, as seems reasonably likely, the economies of certain non-WIOD countries have shapes that are fundamentally different from those of the WIOD countries, then the method described could never be expected to realistically predict their IO tables.

This concern illustrates an important point on how the results from this work should be interpreted. No claim can reasonably made that the estimated IO tables for non-WIOD countries are in any way reliable predictions of the true IO tables underpinning their economies. Instead, for a given non-WIOD country in a particular year, the method should produce an estimated IO table that: (a) is constrained to match certain economic statistics of the true economy of the country in that year, and (b) has a plausible shape, consistent with the shapes of the known IO tables of WIOD countries in the period 1995-2009.





\subsection{Adapting the method}

The method of Section \ref{PCAestimation} has the drawback that it may inappropriately predict negative values for some entries in the IO table. While certain entries of the IO table may be negative, they are generally non-negative, and the method must therefore be altered to ensure that this condition is respected.

To the equality constraints summarised in (\ref{constraints}) are therefore added the inequality constraints:
\begin{equation} \label{ineqconstraints}
x_{i} \geq 0 \; , \; \; \forall i \in \{ 1, \mathellipsis, 160 \}
\end{equation}

In general, there will be no point in $H$ that satifies constraints (\ref{constraints}) and (\ref{ineqconstraints}). The problem therefore becomes one of optimisation. One consequence of this change of perspective is that there is no longer any need to choose a specific number of eigenvectors to ensure that there is a unique intersection point between the PCA hyperplane and the constraint surface, since such a point would not be guaranteed to lie within the feasible region in any case. The number of eigenvalues can therefore be chosen to best represent the apparent dimensionality of the data through consideration of the eigenspectrum.

The complete optimisation problem is defined as follows:

\paragraph*{Quadratic optimisation: Estimating unknown IO tables}
\begin{itemize}
\item Given: $$ H = \left\{ \overline{\textbf{x}} + V \mbox{\boldmath$\alpha$} : \mbox{\boldmath$\alpha$} \in \mathbb{R}^{k} \right\} $$
where:
\begin{equation*}
V = \left(
\begin{tabular}{cccc}
$\uparrow$ & $\uparrow$ & $\mathellipsis$ & $\uparrow$\\
$\textbf{v}^{(1)}$ & $\textbf{v}^{(2)}$ & $\mathellipsis$ & $\textbf{v}^{(k)}$ \\
$\downarrow$ & $\downarrow$ & $\mathellipsis$ & $\downarrow$
\end{tabular} \right) \\
\end{equation*}
for some $k \in \mathbb{N}$.
\item Minimise the distance from $\textbf{x}$ to $H$ in $\mathbb{R}^{160}$.
\item Over: $$\textbf{x} \in \mathbb{R}^{160}$$
\item Subject to:
\begin{align*}
\begin{array}{rl}
A \textbf{x} & = \mbox{\boldmath$\beta$} \\
-x_{i} & \leq 0 \; , \; \; \forall i \in \{ 1, \mathellipsis, 160 \}
\end{array}
\end{align*}
for some $A \in $
\end{itemize}

The objective function can be formulated analytically by introducing the matrix:
$$M = (I - V V^\top)^2$$
giving the optimisation programme:

\paragraph*{Quadratic optimisation: Estimating unknown IO tables}
\begin{itemize}
\item Given: $$ H = \left\{ \overline{\textbf{x}} + V \mbox{\boldmath$\alpha$} : \mbox{\boldmath$\alpha$} \in \mathbb{R}^{k} \right\} $$
\item Minimise: $$\textbf{x}^\top M \textbf{x} - 2 \overline{\textbf{x}}^\top M \textbf{x}$$
\item Over: $$\textbf{x} \in \mathbb{R}^{160}$$
\item Subject to:
\begin{align*}
\begin{array}{rl}
A \textbf{x} & = \mbox{\boldmath$\beta$} \\
-x_{i} & \leq 0 \; , \; \; \forall i \in \{ 1, \mathellipsis, 160 \}
\end{array}
\end{align*}
\end{itemize}


%\begin{figure*}[ht]
%\begin{center}
%\includegraphics[width=0.9\textwidth]{Eigenspectrum_Example.pdf}
%\end{center}
%\captionsetup{width=0.9\textwidth}
%\caption{Eigenspectrum for 1995}
%\label{FIGLABEL}
%\end{figure*} 

\newpage





\section{A Global Demonstration Model}
The model presented here can be described as a network of networks. There is a \textit{primary network}, each node of which entirely contains a \textit{secondary network}. Onto these nested networks are added a small set of simple assumptions which describes how each secondary network responds to changes in its parent node in the primary network, and vice versa. We will now outline the primary and secondary networks.



\subsection{A Network of Networks}
Countries participate in a global trade network wherein a number of goods\footnote{strictly, the output of economic sectors. Each sector will produce a huge array of different goods, but these are treated homogenously throughout this treatment, and the term `good' will be used interchangeably with `sector output'.} are imported and exported in a set of bilateral pairs.

The primary network of the demonstration model has nodes consisting of countries, with each node having a `size' proportional to the GDP of the country. The edges of this primary network are then the bilateral trade relationships between countries. These edges are directed (\textit{who sells to whom?}), weighted (\textit{how much is sold?}) and categorised (\textit{which product is sold?}). Any pair of nodes can thus be associated with many weighted edges, up to a maximum of twice the number of goods in the system.

Within each country, a number of economic sectors---agriculture, raw materials, and manufacturing are examples---participate in a network of sales relationships whereby each sector buys the products of a number of other sectors in order to produce its good. The nodes in the secondary networks are therefore the economic sectors of the country in question, and the directed, weighted and categorised edges are the sales relationships between the sectors. This is an explicit network description the network which is implicitly described by standard input-output modelling.

A network alone, however complex in structure it may be, does not constitute a model of anything. Onto the network, the modeller must attach rules which govern how the network responds to changes in the weight of a particular edge, or the size of a particular node. It is in this spirit which we introduce the three sets of parameters which will govern the behaviour of our network of networks: \textit{technical coefficients}, \textit{trade propensities} and \textit{import ratios}





\subsection{Technical Coefficients, Trade Propensities and Import Ratios}

The network of trade relationships as it currently exists in the real world has, encoded within it, all the unknowable subtleties of international relations and political history. The levels of trade between countries is decided by historical special relationships, cultural ties, physical distances, openness to or tarrifs on trade among many other factors. For example, France may trade more with its former African colonies than it does with other African countries. Previous economic studies have sought to model this explicitly by including a `Franc zone' dummy in regressions. If this phenomenon really exists, it should be reflected in the observed trade data.

Accordingly, the extent to which a country imports a particular product from another country is observed in the data and encoded into the calibration of the model in the form of an \textit{trade propensity}, $p$. Each country has an trade propensity for every product and every possible trading partner. If two countries trade none of a given product, then the corresponding trade propensity is zero. Conversely, if a country imports a product from only one other, the importer's trade propensity will be one.

Observable from the input-output table, is the extent to which a country demands imported products in favour of the equivalent domestically produced product. A country's total import of a product as a ratio of its total demand for that product can therefore be found from data, and encoded in the model as the \textit{import ratio}, $i$, for the product in that country.

By assuming that these three sets of parameters---technical coefficients, trade propensities and import ratios---remain fixed, we can trace the effect on every network edge of an arbitrary change in the weight of any edge in either of the networks.





\pagebreak





\subsection{A complete mathematical description of the demonstration model}





\subsubsection{Parameters and variables of the model} \label{params_vars_of_model}

\noindent The following are parameters of the model: \vspace{2mm}

\begin{tabular}{rcl}
$c$ & : & Number of countries to be considered \vspace{2mm} \\
$s$ & : & Number of economic sectors to be considered \vspace{2mm} \\
\end{tabular}

\noindent The following quantities all represent monetary amounts (in \$): \vspace{2mm}

\begin{tabular}{rcl}
$f^{(Q)}_j$ & : & Final demand on sector $j$ in country $Q$ \vspace{2mm} \\
$n^{(Q)}_j$ & : & Investment of sector $j$ in country $Q$ \vspace{2mm} \\
$e^{(Q)}_j$ & : & Exports of sector $j$ in country $Q$ \vspace{2mm} \\
$x^{(Q)}_j$ & : & Production of sector $j$ in country $Q$ \vspace{2mm} \\
$i^{(Q)}_j$ & : & Imports from sector $j$ of country $Q$ \vspace{2mm} \\
$v^{(Q)}_j$ & : & Value added by sector $j$ in country $Q$ \vspace{2mm} \\
$b^{(Q)}_{j,k}$ & : & Total demand of sector $k$ on sector $j$ in country $Q$ \vspace{2mm} \\
$y^{(R,Q)}_{j}$ & : & Trade flow of sector $j$ from country $R$ to country $Q$ \vspace{2mm} \\
$G^{(Q)}$ & : & GDP of country $Q$ \vspace{2mm}
\end{tabular}

\noindent The corresponding column vectors $\textbf{\textit{f}}^{(Q)},\textbf{\textit{n}}^{(Q)},\textbf{\textit{e}}^{(Q)},\textbf{\textit{x}}^{(Q)},\textbf{\textit{i}}^{(Q)},\textbf{\textit{v}}^{(Q)}$ and matrices $B^{(Q)} = \left( b^{(Q)}_{j,k} \right)$ collect the corresponding quantities for all sectors. \\

\noindent In addition, the column vectors $\textbf{\textit{e}}_j$ and $\textbf{\textit{i}}_j$ respectively collect the exports and imports of sector $j$ from all countries, while the matrix $Y_j = \left( y^{(R,Q)}_{j} \right)$ collects the trade flows of sector $j$ between all countries. \\

\noindent When the symbols $\dag$ and $*$ are appended to a label, the meaning is modified to indicate that part of the relevant value or values that is supplied exclusively through domestic production or exclusively through imports respectively. \\

\noindent The following variables are non-dimensional proportions: \vspace{2mm}

\begin{tabular}{rcl}
$a^{(Q)}_{j,k}$ & : & Demand of sector $k$ on sector $j$ in country $Q$ per unit produced (technical coefficients) \vspace{2mm} \\
$d^{(Q)}_j$ & : & Proportion of the total supply of sector $j$ in country $Q$ that is provided by imports. \vspace{2mm} \\
$p^{(R,Q)}_{j}$ & : & Proportion of $i^{(Q)}_j$ that country $Q$ imports from country $R$ (trade propensities) \vspace{2mm}
\end{tabular}

\noindent $A^{(Q)} = \left( a^{(Q)}_{j,k} \right)$ is the matrix of technical coefficients of country $Q$ for all sectors, $D^{(Q)} = \left( \delta_{ij} d^{(Q)}_j \right)$ is a diagonal matrix containing the values $d^{(Q)}_j$ and $P_j = \left( p^{(R,Q)}_{j} \right)$ is the matrix that collects the trade propensities of sector $j$ for all countries.





\subsubsection{National input-output tables in the model}

\noindent The array $T^{(Q)}$ represents \textbf{the national input-output table} for country $Q$. It collects some of the above variables to present a description of the structure of the economy of country $Q$. \vspace{2mm}

\noindent Neglecting the superscript $(Q)$ for clarity, the array is defined as follows:
$$
\begin{array}{rcc}
T & = & \begin{array}{rc}
\begin{array}{cc} \hspace*{17mm} & \mbox{Sector} \end{array} & \begin{array}{ccccccc} \; \; \; \; 1 \; \; \; \; & \mathellipsis & \; \; s \; \; & \mbox{F.D.} & \mbox{Inv} & \mbox{Exp} & \mbox{Tot} \end{array} \vspace*{2mm} \\
\begin{array}{r}
\begin{array}{rc}
\mbox{Domestic} & \left\{ \begin{array}{c}
1 \\
\vdots \\
s
\end{array} \right. \\
\mbox{Imports} & \left\{ \begin{array}{c}
1 \\
\vdots \\
s \\
\end{array} \right.
\end{array} \vspace*{2mm} \\
\begin{array}{cc} \hspace*{17mm} & \mbox{Value Added} \end{array}
\end{array} &
\left( \begin{array}{ccccccc}
b^{\dag}_{1,1} & \mathellipsis & b^{\dag}_{1,s} & f^\dag_{1} & n^\dag_{1} & e^\dag_{1} & x_1 \\
\vdots & \ddots & \vdots & \vdots & \vdots & \vdots & \vdots \\
b^{\dag}_{s,1} & \mathellipsis & b^{\dag}_{s,s} & f^\dag_{s} & n^\dag_{s} & e^\dag_{s} & x_{s} \\
b^*_{1,1} & \mathellipsis & b^*_{1,s} & f^*_{1} & n^*_{1} & e^*_{1} & i_1 \\
\vdots & \ddots & \vdots & \vdots & \vdots & \vdots & \vdots \\
b^*_{s,1} & \mathellipsis & b^*_{s,s} & f^*_{s} & n^*_{s} & e^*_{s} & i_{s} \vspace*{2mm} \\
v_{1} & \mathellipsis & v_{s} & 0 & 0 & 0 & G
\end{array} \right)
\end{array}
\end{array}
$$

\noindent In matrix form, $T$ may be written:
$$
\begin{array}{rcc}
T & = & \begin{array}{rc}
\begin{array}{cc} \hspace*{17mm} & \mbox{Sector} \end{array} & \begin{array}{ccccccc} 1 & \mathellipsis & s & \mbox{F.D.} & \mbox{Inv} & \mbox{Exp} & \mbox{Tot} \end{array} \vspace*{2mm} \\
\begin{array}{r}
\begin{array}{rr}
\mbox{Domestic} & \left\{ \begin{array}{c}
1 \\
\vdots \\
s
\end{array} \right. \\
\mbox{Imports} & \left\{ \begin{array}{c}
1 \\
\vdots \\
s \\
\end{array} \right.
\end{array} \vspace*{2mm} \\
\begin{array}{cc} \hspace*{17mm} & \mbox{Value Added} \end{array}
\end{array} &
\left( \begin{array}{ccccccc}
 &  &  & \uparrow & \uparrow & \uparrow & \uparrow \\
 & B^{\dag} & & \textbf{\textit{f}}^\dag & \textbf{\textit{n}}^\dag & \textbf{\textit{e}}^\dag & \textbf{\textit{x}} \\
 &  &  & \downarrow & \downarrow & \downarrow & \downarrow \\
 &  &  & \uparrow & \uparrow & \uparrow & \uparrow \\
 & B^* & & \textbf{\textit{f}}^* & \textbf{\textit{n}}^* & \textbf{\textit{e}}^* & \textbf{\textit{i}} \\
 &  &  & \downarrow & \downarrow & \downarrow & \downarrow \vspace*{2mm} \\
\leftarrow & \textbf{\textit{v}}^\top & \rightarrow & 0 & 0 & 0 & G
\end{array} \right)
\end{array}
\end{array}
$$





\subsubsection{The country-level model} \label{countrymod}

\noindent Here, as in the previous section, the superscript $(Q)$ is neglected for clarity. \\

\noindent A key assumption of the model is that, for a given country and for a given sector, imports constitute a fixed proportion of total supply (the sum of imports and domestic production), as described by the equation:
\begin{align} \label{di_assumption}
i_{j} = d_{j} ( x_{j} + i_{j} )
\end{align}
Or in vector form:
\begin{align} \label{di_assumption_v}
\textbf{\textit{i}} = D ( \textbf{\textit{x}} + \textbf{\textit{i}} )
\end{align}

\noindent The overall final demand, investment and exports for each sector are the sum of a domestic component and an imported component:
$$
\begin{array}{rclcrcl}
f_j &=& f_j^\dag + f_j^* & \; \; \;  & \textbf{\textit{f}} &=& \textbf{\textit{f}}^\dag + \textbf{\textit{f}}^* \\
n_j &=& n_j^\dag + n_j^* & \; \; \; & \textbf{\textit{n}} &=& \textbf{\textit{n}}^\dag + \textbf{\textit{n}}^* \\
e_j &=& e_j^\dag + e_j^* & \; \; \; &\textbf{\textit{e}} &=& \textbf{\textit{e}}^\dag + \textbf{\textit{e}}^*
\end{array}
$$

\noindent In practice, the division of these quantities according to whether they are supplied domestically or through imports is not relevant to the model and only the aggregated quantities are considered. In any case in which it is necessary to separate such quantities in this way, it is assumed that they are divided in the same ratio as are the total domestic production and imports of the corresponding sector: \\
$$
\begin{array}{rclcrcl}
f^*_j &=& D \textbf{\textit{f}}  \\
n^*_j &=& D \textbf{\textit{n}}  \\
e^*_j &=& D \textbf{\textit{e}}  
\end{array}
$$

\noindent For a particular country, in order for supply to match demand, the following equality must be fulfilled for each sector $j$:
\begin{align}
x_j + i_{j} &= f_{j} + n_{j} + e_{j} + \displaystyle \sum_{k=1}^{s} x_k a_{j,k}
\end{align}
In vector form, this becomes:
\begin{align} \label{supply_demand_eqn}
\textbf{\textit{x}} + \textbf{\textit{i}} &= \textbf{\textit{f}} + \textbf{\textit{n}} + \textbf{\textit{e}} + A \textbf{\textit{x}}
\end{align}

\noindent Given the final demand, investment and exports of all sectors for a particular country, we would like to use this equation to determine the necessary production and import totals for each sector. In other words, given $A$, $\textbf{\textit{f}}$, $\textbf{\textit{n}}$, $\textbf{\textit{e}}$, we wish to solve (\ref{supply_demand_eqn}) for $\textbf{\textit{x}}$ and $\textbf{\textit{i}}$. Taken together, $\textbf{\textit{x}}$ and $\textbf{\textit{i}}$ contain $2s$ separate variables, while (\ref{supply_demand_eqn}) represents a system of $s$ equations. However, the assumption of (\ref{di_assumption_v}) reduces the number of independent variables to $s$, allowing for the possibility of a unique solution. \\

\noindent (\ref{di_assumption_v}) can therefore be rearranged and rewritten as:

\begin{align}  \label{supply_demand_eqn_1}
\left( I - (I-D)A \right) \textbf{\textit{x}} &= ( I-D ) \left( \textbf{\textit{f}} + \textbf{\textit{n}} + \textbf{\textit{e}} \right)
\end{align}
where $I$ is an identity matrix of the appropriate size. \\

\noindent Provided that $\left( I - (I-D)A \right)$ has an inverse, given $A$, $D$, $\textbf{\textit{f}}$, $\textbf{\textit{n}}$, $\textbf{\textit{e}}$, (\ref{supply_demand_eqn_1}) can be solved for $\textbf{\textit{x}}$.





\subsection{Global goods trade}  \label{goodstrade}

\noindent A second key assumption of the model is that for each pair of countries $(R,Q)$ and for each sector $j$, the imports $y^{(R,Q)}_{j}$ of country $Q$ that are supplied by country $R$ is a fixed proportion $p^{(R,Q)}_{j}$ (the \textbf{trade propensity}) of the total imports $i^{(Q)}_j$ of country $Q$ for that sector, as described by the equation:
\begin{align} \label{import_propensity_eqn}
y^{(R,Q)}_{j} = p^{(R,Q)}_{j} i^{(Q)}_j
\end{align}
Observe that:
\begin{align}
\displaystyle \sum_{Q=1}^{c} y^{(R,Q)}_{j} = e^{(R)}_j
\end{align}
In matrix form, (\ref{import_propensity_eqn}) may therefore be rewritten as:
\begin{align} \label{find_exports}
\textbf{\textit{e}}_j = P_{j} \textbf{\textit{i}}_j
\end{align}

\noindent Observe that, given import vectors $\textbf{\textit{i}}^{(Q)}$ for all countries $Q$, the import quantities $i^{(Q)}_j$ can be rearranged into import vectors $\textbf{\textit{i}}_j$ for all sectors $j$. Equation (\ref{find_exports}) can then be used to find export vectors $\textbf{\textit{e}}_j$ for all sectors $j$ and the export quantities $e^{(R)}_j$ can again be rearranged into export vectors $\textbf{\textit{e}}^{(R)}$ for all countries $R$.





\clearpage





\noindent Import and export matrices $M$ and $E$ are defined:
\begin{align*}
M = \left(
\begin{array}{ccc}
\uparrow & \hdots & \uparrow \\
{\textbf{\textit{i}}^{(1)}} & \hdots & {\textbf{\textit{i}}^{(c)}} \\
\downarrow & \hdots & \downarrow
\end{array} \right)
\\
E = \left(
\begin{array}{ccc}
\uparrow & \hdots & \uparrow \\
{\textbf{\textit{e}}^{(1)}} & \hdots & {\textbf{\textit{e}}^{(c)}} \\
\downarrow & \hdots & \downarrow
\end{array} \right)
\end{align*}





\subsection{Model algorithm} \label{model_algorithm}

The country level model and the model of global goods trade may be combined to create a comprehensive model of global trade both within and between national economies. This model is based on the following algorithm. Note that the algorithm involves an iterative procedure, where variables followed by square brackets represent the value of the variable at the appropriate iteration.

\paragraph{Step 0: Initialisation} For all countries $Q$ and for all sectors $j$, $A^{(Q)}$, $D^{(Q)}$, $\textbf{\textit{f}}^{(Q)}$, $\textbf{\textit{n}}^{(Q)}$ and $P_j$ are exogenous. They may be derived from data or set in some other way. Initially, it is assumed that there are no exports: $\textbf{\textit{e}}^{(Q)}[0] = \mathbf{0} \; , \; \; \forall Q \in \{1, \mathellipsis , c \}$. The iteration variable $t$ is set to zero and a small tolerance value $\epsilon > 0$ is chosen.

\paragraph{Step 1: Calculation of production vectors} For each country $Q$, (\ref{supply_demand_eqn_1}) is solved with $A^{(Q)}$, $D^{(Q)}$, $\textbf{\textit{f}}^{(Q)}$, $\textbf{\textit{n}}^{(Q)}$ and $\textbf{\textit{e}}^{(Q)}[t]$ to produce a new estimate of the production vector $\textbf{\textit{x}}^{(Q)}[t]$.

\paragraph{Step 2: Calculation of import vectors} For each country $Q$, (\ref{di_assumption_v}) is solved with $D^{(Q)}$ and $\textbf{\textit{x}}^{(Q)}[t]$ to produce a new estimate of the import vector $\textbf{\textit{i}}^{(Q)}[t]$. The import quantities $i^{(Q)}_j[t]$ may also be rearranged into import vectors $\textbf{\textit{i}}_j[t]$ for all sectors $j$.

\paragraph{Step 3: Condition for continued iteration} Consider the following inequality:
\begin{align} \label{tolerance_equation}
\left| \sum_{Q=1}^{c} \left(\textbf{\textit{i}}^{(Q)}[t] - \textbf{\textit{e}}^{(Q)}[t] \right) \right| \geq \epsilon
\end{align}

In a consistent system, for a given sector, the sum of the imports to all countries should be equal to the sum of the exports from all countries. Therefore, if (\ref{tolerance_equation}) holds, the algorithm is considered not to have converged and the process continues from \textit{Step 4}. Otherwise, the process skips to \textit{Step 6}.

\paragraph{Step 4: Calculation of export vectors} For each sector $j$, (\ref{find_exports}) is solved with $P_j$ and $\textbf{\textit{i}}_j[t]$ to produce a new estimate of the export vector $\textbf{\textit{e}}_j[t+1]$. The new export quantities $e^{(R)}_j[t+1]$ may also be rearranged into export vectors $\textbf{\textit{e}}^{(R)}[t+1]$ for all countries $R$.

\paragraph{Step 5: End of iteration} The iteration variable $t$ is increased by $1$ and the process returns to \textit{Step 1}.

\paragraph{Step 6: Termination} The algorithm terminates, returning $\textbf{\textit{x}}^{(Q)} = \textbf{\textit{x}}^{(Q)}[t]$, $\textbf{\textit{i}}^{(Q)} = \textbf{\textit{i}}^{(Q)}[t]$, $\textbf{\textit{e}}^{(Q)} = \textbf{\textit{e}}^{(Q)}[t]$, for all $Q \in \{ 1, \mathellipsis, c \}$.





\subsection{Callibration of model quantities from data}

In this section, we assume that data is available providing complete and consistent information on the following (in \$):

\begin{itemize}
\item Final demand on each sector in each country;
\item Investment of each sector in each country;
\item Total production of each sector in each country;
\item Inter-sector flows within each country (as given in an input-output table), separated into those supplied by imports and those supplied by domestic production;
\item Trade flows from each sector of each country to each sector of each other country.
\end{itemize}

In reality, of course, many of these quantities may not be known or their values as given in a particular dataset may not be consistent. For example, total world imports may not be equal to total world exports. In practice therefore, for a given dataset, certain of these quantities will have to be estimated or adjusted. Specific issues presented by the data used in this article are discussed in Sections SECTIONS.

In terms of the variables defined in Section \ref{params_vars_of_model}, we have that the following quantities are known, for all $Q,R \in \{1, \mathellipsis, c \}$, $j,k \in \{1, \mathellipsis, s \}$:

$$
\overline{f}^{(Q)}_j \; \; , \; \; \overline{n}^{(Q)}_j \; \; , \; \; \overline{x}^{(Q)}_j \; \; , \; \; \overline{b}^{\dag (Q)}_{j,k} \; \; , \; \; \overline{b}^{* (Q)}_{j,k} \; \; , \; \; \overline{y}^{(R,Q)}_{j}
$$

Note that the bar on each of these variables indicates that we are considering an observed value of the corresponding quantity, taken directly from data, rather than a variable of the model. Clearly, other monetary amounts, such as $\overline{e}^{(Q)}_j$, $\overline{i}^{(Q)}_j$, $\overline{v}^{(Q)}_j$, $\overline{b}^{(Q)}_{j,k}$ and $\overline{G}^{(Q)}$, can be deduced directly from such data.

The non-dimensional parameters of the model are derived from these known quantities as follows:

\begin{align} \label{callibrate_params_from_data}
\begin{array}{rcc}
a^{(Q)}_{j,k} &=& \displaystyle \frac{\overline{b}^{(Q)}_{j,k}}{\overline{x}^{(Q)}_j} \vspace{3mm} \\
d^{(Q)}_j &=& \displaystyle \frac{\overline{i}^{(Q)}_j}{ \overline{x}^{(Q)}_j + \overline{i}^{(Q)}_j} \vspace{3mm} \\
p^{(R,Q)}_{j} &=& \displaystyle \frac{\overline{y}^{(R,Q)}_{j}}{\overline{i}^{(Q)}_j}
\end{array}
\end{align}





\subsection{Callibration of a `Rest of world' entity}

In some situations, it may be necessary or desirable to group a number of countries together as a single entity known as `Rest of world', represented by the country index $Q=0$. This will be the case if we wish to focus only on a certain set of countries; for instance, because the quality of available data is higher for these countries than for others.

In matters of international trade (Section \ref{goodstrade}), `Rest of World' behaves in an identical way to individual countries. However, the rules governing the internal behaviour of its economy are different (Section \ref{goodstrade}). In terms of the algorithm of Section \ref{model_algorithm}, this means that `Rest of World' is treated as a model country at \textit{Step 4}, but is updated in a unique way at \textit{Steps 1} and \textit{2}.

Suppose that, from data, the following quantities are known for all sectors $j,k \in \{1, \mathellipsis, s \}$, but for only a certain subset of all countries $Q,R \in \{1, \mathellipsis, \tilde{c} \}$ (where countries have been relabelled for convenience):
$$
\overline{f}^{(Q)}_j \; \; , \; \; \overline{n}^{(Q)}_j \; \; , \; \; \overline{x}^{(Q)}_j \; \; , \; \; \overline{b}^{\dag (Q)}_{j,k} \; \; , \; \; \overline{b}^{* (Q)}_{j,k} \; \; , \; \; \overline{y}^{(R,Q)}_{j} \; \; , \; \; \overline{e}^{(Q)}_j \; \; , \; \; \overline{i}^{(Q)}_j
$$

Note that although we assume that trade flows $\overline{y}^{(R,Q)}_{j}$ are only known between countries in the given subset, the known values of total imports $\overline{i}^{(Q)}_j$, and total exports $\overline{e}^{(Q)}_j$ for these countries include trade with \textit{all} countries.

Observe that, for a given country $Q \in \{1, \mathellipsis, \tilde{c} \}$ and a given sector $j \in \{1, \mathellipsis, s \}$, trade flows to and from `Rest of world' may be determined as follows:

\begin{itemize}
\item The value of exports from $Q$ to `Rest of world' equals the total value of exports from $Q$ for which the destination is not specified in the available data:
$$
y^{ \left( Q,0 \right) }_{j} = \overline{e}^{(Q)}_j - \sum_{R=1}^{\tilde{c}} \overline{y}^{(Q,R)}_{j}
$$
\item The value of imports to $Q$ from `Rest of world' equals the total value of imports to $Q$ for which the source is not specified in the available data:
$$
y^{ \left( 0,Q \right) }_{j} = \overline{i}^{(Q)}_j - \sum_{R=1}^{\tilde{c}} \overline{y}^{(R,Q)}_{j}
$$
\end{itemize}

For each sector $j$, the total imports and exports of `Rest of world' may therefore be determined by simply summing over all countries:
\begin{align*}
i^{(0)}_j &= \sum_{Q=1}^{\tilde{c}} y^{ \left( Q,0 \right) }_{j} \\
e^{(0)}_j &= \sum_{Q=1}^{\tilde{c}} y^{ \left( 0,Q \right) }_{j}
\end{align*}

Given these quantities, the trade propensities $p^{ \left( Q,0 \right) }_j$ and $p^{ \left( 0,Q \right) }_j$, for all sectors $j$ and countries $Q$, may be calculated using (\ref{callibrate_params_from_data}).

`Rest of world' does not have a corresponding input-ouput table, since reasonable import ratios and technical coefficients could not be determined for it with the available data. However, in order to fulfil its role in world trade, `Rest of world' must still demand and produce products and services in the various economic sectors. To do this, the following assumptions are made about various quantities of the model:

\begin{itemize}
\item All imports of `Rest of world' are used to directly supply final demand:
$$
\textbf{\textit{f}}^{(0)} = \textbf{\textit{i}}^{(0)}
$$
\item Investments of `Rest of world' are equal to zero.
$$
\textbf{\textit{n}}^{(0)} = \textbf{0}
$$
\item Everything produced in `Rest of world' is exported.
$$
\textbf{\textit{x}}^{(0)} = \textbf{\textit{e}}^{(0)}
$$
\end{itemize}

Since `Rest of world' does not have import ratios or technical coefficients, any demand on `Rest of world' for exports at \textit{Step 1} of the model algorithm (Section \ref{model_algorithm}) is automatically directly fulfilled by increased production $\textbf{\textit{x}}^{(0)}[t] = \textbf{\textit{e}}^{(0)}[t]$, while at \textit{Step 2}, import requirements are always set equal to final demand $\textbf{\textit{i}}^{(0)}[t] = \textbf{\textit{f}}^{(0)}[t]$. In this way, at every iteration of the model algorithm, `Rest of world' is both a direct sink for excess exports (through final demand) and a direct source of unsupplied imports (through production) for the model countries.



























\pagebreak





\subsection{Global services trade} \label{servicestrade}
Data on trade in goods is based on the very detailed records kept by border agencies for gathering the appropriate taxes. For this reason, the goods trade data is considered to be of high quality. Since the supply of services is harder to track, the data on trade in service goods is considerably coarser. Specifically, many countries report only the quantity of services imported and exported, not the origin and destination of any particular bilateral trade.





\subsubsection{Accounting constraints} \label{accountingconstraints}
To estimate the actual trade flows between countries, we start by imposing some simple accounting restrictions on the estimates. The trade in the good of sector $j$ from country $R$, the exporter, to country $Q$, the importer is given by $y_j^{(R,Q)}$. For all sectors, we must have:
\begin{equation}
\sum\limits_{Q} y_j^{(R,Q)} = e_j^{R}
\end{equation}
such that the flows of sector $j$ from country $R$ to all destinations is equal to country $R$s export of sector $j$, and
\begin{equation}
\sum\limits_{R} y_j^{(R,Q)} = i_j^{Q}
\end{equation}
such that the flows of sector $j$ into country $Q$ from all origins is equal to country $Q$s import demand for sector $j$. Finally, in order for the flow of goods to be a closed system globally, we need:
\begin{equation} \label{eqn:ImportsMustMatchExports}
\sum\limits_{Q} i_j^{Q} = \sum\limits_{Q} e_J^{(Q)}
\end{equation}

We specify that a country cannot satisfy any of its import demand using its own exports, such that:
\begin{equation} \label{noselfimports}
y_j^{(Q,Q)} = 0 \qquad \forall \; j,Q
\end{equation}

There exist simple and well-understood methods for dealing with this kind of problem, two of which are outlined below.





\subsubsection{Iterative proportional fitting} \label{iterativeproportionalfitting}
Consider a generalised table of double-indexed variables, $y_{ij}$. In the present example, the $y_ij$s would represent services flows between country $i$, the exporter, and country $j$, the importer. By placing the countries in an arbitrary, but consistent, order these flows can be represented as a square matrix as follows:
\begin{equation}
\textbf{Y} = \left(
\begin{array}{ccc}
y_{11} & \dots & y_{1J} \\
\vdots & \ddots & \vdots \\
y_{I1} & \dots & y_{IJ} \\
\end{array}
\right)
\end{equation}
where $I$ and $J$ are the total number of rows and columns respectively. Let the row totals, $r_j = \sum\limits_i y_{ij}$, and the column totals, $c_i = \sum\limits_{j} y_{ij}$, be known for all $i$ and $j$ and all $y_{ij}$s be unknown\footnote{with the possible exception of some $y_{ij}$s being forced to zero. See equation \ref{noselfimports} for the importance of this for the current application.}.

The Iterative proportional fitting procedure (IPFP) is a simple algorithm to produce a table of numbers whose row sums equal $r$ and whose column sums equal $c$.IPFP requires the iteration of a simple weighting procedure, scaling first the rows, then the columns, to sum to the required row and column totals, until the row and column sums both are within a certain $\epsilon$, arbitrarily small, of the required values. Each iteration produces an estimated matrix $\textbf{Y}^{(n)}$ where $n$ is the number of the iteration. The algorithm is as follows:
\begin{enumerate}
\item set all elements of $\textbf{Y}^{(0)}$ to some initial value, e.g. unity \label{initialise}
\item calculate the row sum of the first row of $\textbf{Y}^{(0)}$, denoted $\hat{r}_1$\label{rowsum1}
\item multiply each element of the first row of $\textbf{Y}^{(0)}$ by $\dfrac{r_1}{\hat{r}_1}$ to ensure that the first row sums to exactly $r_1$. \label{rowsum2}
\item repeat steps \ref{rowsum1} and \ref{rowsum2} for all rows \label{rowsum3}
\item repeat steps \ref{rowsum1}, \ref{rowsum2} and \ref{rowsum3} for the columns of $\textbf{Y}^{(0)}$, replacing $r_i$ and $\hat{r}_i$ with $c_j$ and $\hat{c}_j$ in the obvious way. This results in a new matrix $\textbf{Y}^{(1)}$ \label{colsums}
\item repeat steps \ref{rowsum1}, \ref{rowsum2}, \ref{rowsum3} and \ref{colsums} until $|{r_i - \hat{r}_i}| < \epsilon$ and $|c_j - \hat{c}_j| < \epsilon$
\end{enumerate}
 
The row and column sums can be shown to converge to the desired values arbitrarily closely for any matrix $\textbf{Y}$ including one where some of the elements are initially set to zero. Since the algorithm involves nothing but multiplying and dividing the elements of $\textbf{Y}$ any elements set to zero will remain zero throughout the procedure.

In the case of import/export flows between countries, the IPFP will split each sector's
total flows into flows to/from countries in proportion to the extent that each country imports/exports that sector's good. For example, if the USA is the largest exporter of financial services, all countries will be assumed to satisfy the largest part of their import demand with US goods. Similarly with imports, IPFP will model a country as exporting most to those countries which import the most of a given good.

This may or may not be an appropriate modelling assumption. Perhaps a more realistic framework is one where countries' propensities to trade in goods with one another influences their propensities to trade in services. For example, India may be a big exporter of services, but if Pakistan has very low goods trade with India, we might expect it to have a similarly low services trade. A method for dealing with this is described in the next section.





\subsubsection{Entropy maximisation}
The procedure described in the subsection above is guaranteed to produce a feasible solution to the problem of distributing total flows into bilateral flows between countries. But the result is defined purely by considerations of volume: importers receive goods from those countries which produce the most and, conversely, exporters sell the most to those countries which demand the most. While this setup is consistent with a traditional economics viewpoint, there are subtleties in real international trade which are not explained simply by supply and demand.

For example, countries tend to trade more with countries which are closer to them, for the simple reason that transport costs are lower when the physical distances are small. Countries will also tend to trade with strategic partners, countries with low border tariffs or countries with historical trade links. How can we hope to reflect this complexity in a framework which is simple and data-driven? One possible way is to treat existing trade relationships, as reflected in actual trade data, as being a proxy for all the unobserved historical and political reasons for preferring to trade with one country over another. In this spirit, we `re-purpose' the trade propensities matrix, $\textbf{P}$, described in section \ref{goodstrade}, as an impedance matrix in a traditional entropy maximisation. Countries will then favour trading with those with whom they already trade the most. **  MORE TO FOLLOW HERE, NATCH **





\subsubsection{An over-constrained trade model}
The seemingly harmless constraint that no country may satisfy its import demand using its own export, as described by \eqref{noselfimports}, can, in some circumstances, lead to the system being over-constrained. Consider the following table of import/export totals of the good from a single sector for three countries, $X$, $Y$, and $Z$, as shown in table \ref{tab:xyzflows}.

\begin{table}
\centering
\begin{tabular}{llr}
\toprule
Country & Flow direction & Flow amount \\ 
\midrule
$X$ & Import & 20 \\
$X$ & Export & 15 \\ \addlinespace
$Y$ & Import & 2 \\
$Y$ & Export & 7 \\ \addlinespace
$Z$ & Import & 5 \\
$Z$ & Export & 5 \\
\bottomrule
\end{tabular}
\caption{a sample table of total imports and exports of a single good showing how a system may be over-constrained when countries may not satisfy their import demand using their own exports}
\label{tab:xyzflows}
\end{table}

The system is balanced, to the extent that total world imports sum to total world exports with 27 units. But the IPFP will never arrive at a stable solution where the initial conditions of the flow matrix are set such that \eqref{noselfimports} is satisfied, i.e. that the diagonal elements of the initial flow matrix are zero. That no solution to this system exists can be seen by examining the import demand of country $X$. That country requires 20 units of the good, but the combined export of all \textit{other} countries sums to just 12 units, a shortfall of 8 units. If $X$ is to be able to import 20 units of the good, it \textit{must} import from its own export goods, and this is explicitly disallowed by \eqref{noselfimports}.

The problem, of course, stems from the fact that not all the countries in the world feature in our minimal data set. There must be `some other' countries, not present in the data, which export enough to satisfy $X$'s import demand. To model this, we include an additional row and column in all such analyses to represent the rest of the world (RoW). The RoW then behaves exactly as a normal country, with both imports and exports across its borders, but it is considered to have an infinite stock of exports and an infinite demand for import and is allowed to satisfy its import demand with its own exports, in contrast to the other countries, since whatever countries implicitly make up the `rest of the world' are free to trade to any extent with one another. Indeed, by setting the initial value of RoW--RoW trade to be orders of magnitude higher than any other flow, the RoW becomes a `trader of last resort', providing only those goods which cannot be provided by the other countries. In this way, the RoW country serves as both a source and a sink for all goods, ensuring that all import demands can be met and all exports have a destination. The initial flow matrix $\textbf{Y}$ will then look as in figure \ref{fig:ipfpexamplebefore}.

\begin{figure}
\centering
\begin{subfigure}[t]{0.45\textwidth}
\centering
\begin{tabular}{ccccl}
\toprule
      & $X$ & $Y$ & $Z$ & RoW \\ \midrule 
$X$   &  0  &  1  &  1  &  1  \\
$Y$   &  1  &  0  &  1  &  1  \\
$Z$   &  1  &  1  &  0  &  1  \\
RoW   &  1  &  1  &  1  &  $10^8$  \\
\bottomrule
\end{tabular}
\caption{The initial flow matrix, $\textbf{Y}^0$, for resolving potentially over-constrained import/export totals using a `rest of the world' (RoW) country. Notice that the RoW is the only country allowed to satisfy its import demand from its own exports.}
\label{fig:ipfpexamplebefore}
\end{subfigure}
\qquad % horizontal spacing
\begin{subfigure}[t]{0.45\textwidth}
\centering
\begin{tabular}{ccccl}
\toprule
      & $X$ & $Y$ & $Z$ & RoW \\ \midrule 
$X$   &  0.0  &  2.0  &  5.0  &  8.0  \\
$Y$   &  7.0  &  0.0  &  0.0  &  0.0  \\
$Z$   &  5.0  &  0.0  &  0.0  &  0.0  \\
RoW   &  8.0  &  0.0  &  0.0  &  9992.0  \\
\bottomrule
\end{tabular}
\caption{The flow matrix, $\textbf{Y}^0$, correct to one decimal place, after balancing with the IPFP. The RoW is used only by $X$ to satisfy the excess import demand which $Y$ and $Z$ together are unable to supply. Both $Y$ and $Z$ import from the majority supplier $X$.}
\label{fig:ipfpexampleafter}
\end{subfigure}
\caption{The iterative proportional fitting procedure with data as given in table \ref{tab:xyzflows}. Figure \ref{fig:ipfpexamplebefore} shows the initial conditions and figure \ref{fig:ipfpexampleafter} the equilibrium outcome.}
\label{fig:RoWexample}
\end{figure}

For implementation, RoW is given a large export total of each good (e.g. $10^8$) and the RoW import demand is then set such that global imports equal global exports. Notice that this alleviates the need for the procedure of scaling exports up to match imports as described in section \ref{accountingconstraints}, since \eqref{eqn:ImportsMustMatchExports} is satisfied by construction.

The results of the IPFP applied to the example data in table \ref{tab:xyzflows} is shown in figure \ref{fig:ipfpexampleafter}. The procedure converged to within $\epsilon=10^{-5}$ in 29 iterations. The outcome of the IPFP is intuitively satisfying. The RoW fulfils its role in providing the 8 unit shortfall to satisfy $X$'s remaining import demand. The other countries import exclusively from $X$, which is the major exporter among the explicitly modelled countries and $X$ completely exhausts the other countries' exports as might be expected with $X$ being so dominant in the market. 





\section{Design Details}
If a country imports nothing of a certain product when the model is being set up, there is no way of calculating the trade propensities. We therefore arbitrarily set the trade propensity for that product to be zero for all exporting countries, and unity for the Rest of the World (RoW) country.
\end{document} 
